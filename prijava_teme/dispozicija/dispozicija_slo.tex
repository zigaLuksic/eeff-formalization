\documentclass{article}
\usepackage[utf8]{inputenc}
\usepackage[T1]{fontenc}
\usepackage{mathptmx}

\usepackage{amsmath}
\usepackage{amssymb}
\usepackage{mathpartir}
\usepackage{stmaryrd}
\usepackage{newtxmath}
\usepackage{ifthen}
\usepackage{xcolor}
\usepackage{epigraph}
\usepackage{tikz-cd}
\usepackage{float}
\usepackage{framed}
\usepackage{caption}
\usepackage{tcolorbox}
\usepackage{enumitem}

\renewcommand{\and}{\\}

\title{Dispozicija doktorske disertacije}
\author{Žiga Lukšič \and Mentor: dr. Matija Pretnar}
\date{}

%===============================================================================
\begin{document}

\maketitle

\vspace{-10mm}
\begin{center}
  \Large{\textsc{\textbf{Kako sosedu ukrasti sendvič brez da opazi}}}

  \Large{\textsc{\textbf{(How to steal your neighbours sandwich without being noticed)}}}
\end{center}

Kot so to opazili že mnogi~\cite{DBLP:conf/icfp/KammarLO13,DBLP:journals/entcs/Pretnar15,DBLP:journals/pacmpl/Ahman18} v splošnem nimam pojma kaj počnem.


\section*{Pregled znanstvenega področja in dosedanjih raziskav}



\section*{Opis raziskovalne vsebine}

Uporaba

\section*{Raziskovalna vprašanja}

Kaj vse lahko natlačiva v Eff preden postane neokusno? Raziščimo s pristopom surove sile.

\section*{Pričakovani rezultati in prispevek k znanosti}

Pričakovani rezultati so kul teorija, ki se bo v dejanskih programskih jezikih pojavila šele po tem, ko Elon pristane na Marsu.

\renewcommand\refname{Literatura}
\bibliographystyle{abbrv}
\bibliography{bibliography}

\end{document}
