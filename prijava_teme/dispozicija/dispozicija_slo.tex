\documentclass{article}
\usepackage[utf8]{inputenc}
\usepackage[T1]{fontenc}
\usepackage{mathptmx}

\usepackage{amsmath}
\usepackage{amssymb}
\usepackage{mathpartir}
\usepackage{stmaryrd}
\usepackage{newtxmath}
\usepackage{ifthen}
\usepackage{xcolor}
\usepackage{epigraph}
\usepackage{tikz-cd}
\usepackage{float}
\usepackage{framed}
\usepackage{caption}
\usepackage{tcolorbox}
\usepackage{enumitem}

\renewcommand{\and}{\\}

\title{Dispozicija doktorske disertacije}
\author{Žiga Lukšič \and Mentor: dr. Matija Pretnar}
\date{}

%===============================================================================
\begin{document}

\maketitle

\vspace{-10mm}
\begin{center}
  \Large{\textsc{\textbf{Kako sosedu ukrasti sendvič brez da opazi}}}

  \Large{\textsc{\textbf{(How to steal your neighbours sandwich without being noticed)}}}
\end{center}

Kot so to opazili že mnogi~\cite{DBLP:conf/icfp/KammarLO13,DBLP:journals/entcs/Pretnar15,DBLP:journals/pacmpl/Ahman18} v splošnem nimam pojma kaj počnem.


\section*{Pregled znanstvenega področja in dosedanjih raziskav}

Uporaba algebrajskih učinkov in prestreznikov je del področja teorije tipov, čigar vpliv na razvoj programskih jezikov hitro narašča. Raziskave področja so se začele predvsem zaradi iskanje matematičnega pomena za računske učinke \cite{DBLP:conf/fossacs/PlotkinP01, DBLP:journals/acs/PlotkinP03}, kasneje pa se je izkazalo, da so algebrajski učinki uporabni tudi kot del samih programskih jezikov \cite{DBLP:journals/corr/PlotkinP13}. Algebrajski učinki nam omogočajo preprosto in generično razširitev funkcijskih programskih jezikov z računskimi učinki. Z uporabo naprednejšega sistema tipov \cite{DBLP:conf/esop/SalehKPS18} zagotavljajo varnost kljub prisotnosti računskih učinkov. Prav tako se lepo sklopijo z obstoječo denotacijsko semantiko funkcijskih programskih jezikov, kar nam ponuja vpogled v matematični pomen programov. 

\section*{Opis raziskovalne vsebine}

Razvoj zanesljivih in ekspresivnih sistemov tipov skrbi tako za varnost programov kot tudi za dokazovanje lastnosti in pravilnosti programske kode. Z uporabo algebrajskih učinkov lahko sistem tipov razširimo na način, ki nam omogoča natančno sledenje računskih učinkov. Hkrati pa nam teorija algebrajskih učinkov nudi tudi enačbe, ki veljajo med učinki. Sistem tipov učinkov tako razširimo na sistem tipov teorij, ki poleg možnih učinkov hranijo tudi enačbe teorije. Te enačbe nam omogočajo lažje dokazovanje lastnosti programske kode, dodatno varnost s pomočjo iskanja napak v kodi, in odpirajo vrata novim pristopom k optimizaciji kode s pomočjo enačb.

\section*{Raziskovalna vprašanja}

Kaj vse lahko natlačiva v Eff preden postane neokusno? Raziščimo s pristopom surove sile.

\section*{Pričakovani rezultati in prispevek k znanosti}

Pričakovani rezultati so kul teorija, ki se bo v dejanskih programskih jezikih pojavila šele po tem, ko Elon pristane na Marsu.

\renewcommand\refname{Literatura}
\bibliographystyle{abbrv}
\bibliography{bibliography}

\end{document}
