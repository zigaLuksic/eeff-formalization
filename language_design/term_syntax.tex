% !TeX root = ./main.tex
% chktex-file 1
% chktex-file 46

\[
	\begin{array}{rrll}
		\text{values}~v
		 & ::=   & x                      			& \text{variable}          \\
		 & \vsep & \unit												& \text{unit}							 \\
		 & \vsep & n														& \text{integer}					 \\
		 & \vsep & \inleft{v}~\vsep~\inright{v} & \text{sum constructors}  \\
		 & \vsep & \pair{v_1}{v_2}							&	\text{pair}						 	 \\
		 & \vsep & \fun{x}{c}             			& \text{function}          \\
		 & \vsep & \handler{\hcases}      			& \text{handler}           \\
		 \\
		 \text{computations}~c
		 & ::=   & \return{v}             		& \text{returned value}    		\\
		 & \vsep & \apply{v_1}{v_2}       		& \text{application}       		\\
		 & \vsep & \match{v} \inleft{x} \mapsto c_1 
		 							\vsep \inright{x} \mapsto c_2 & \text{sum match} 			\\
		 & \vsep & \match{v} \pair{x}{y} \mapsto c 	& \text{product match} 	\\
		 & \vsep & \operation{v}{y.c}     		& \text{operation call}    		\\
		 & \vsep & \letrec{f}{x}{c_1} {c_2}		& \text{recursive function} 	\\
		 & \vsep & \dobind{x}{c_1}{c_2}   		& \text{sequencing}        		\\
		 & \vsep & \withhandle{v}{c}      		& \text{handling}					 		\\
		 \\
		 \text{operation clauses}~\hcases
			& ::=   & \emptyhcases 
			\vsep 		\hcases \union \set{\operation{x}{k} \mapsto c_{\op}} 	\\
		\end{array}
		\]
